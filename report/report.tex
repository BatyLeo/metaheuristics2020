\documentclass[12pt]{article}

% essential packages
\usepackage[utf8]{inputenc}
\usepackage[french]{babel}
\usepackage[T1]{fontenc}

% other packages
\usepackage{packages}
\usepackage{algorithm}
\usepackage{algorithmic}

% Pour le glossaire
% \usepackage[toc, automake, acronym]{glossaries}
% \makeglossaries

\title{Étude de cas : k-couverture connexe minimum dans les réseaux de capteurs}
\author{\textsc{BATY Léo}, \textsc{BRUNOD-INDRIGO Luca}}
\date{3 novembre 2020}

\begin{document}

\maketitle

\tableofcontents

\newpage

\section{Calcul de bornes inférieures}

Dans cette section, on pose le problème sous la forme d'un programme linéaire en nombres entiers (PLNE) et on en déduit des bornes inférieures par relaxation linéaire.

\subsection{Notations}

Voici tout d'abord les différentes notations qui seront utilisées dans la suite :

\begin{bulletlist}
  \item $k\in \{1, 2, 3\}$
  \item $R^{capt} \leq R^{com}$
  \item $T$ ensemble des cibles (targets), de cardinal $n$
  \begin{itemize}
    \item[$\rightarrow$] $t\in T$, coordonnées $(x_t, y_t)$
    \item[$\rightarrow$] $t, t'\in T, t\neq t', \Delta_{t, t'} = \sqrt{(x_t - x_{t'})^2 + (y_t - y_{t'})^2}$
  \end{itemize}
  \item puit $s$ de coordonnées $(x_s, y_s)$
  \item $E^{capt} = \{(t, t')\in T^2 | t\neq t', \Delta_{t, t'} \leq R^{capt}\}$
  \begin{itemize}
    \item[$\rightarrow$] $E^{capt}_t = \{t'\in T | (t, t')\in E^{capt}\}$
    \item[$\rightarrow$] \textbf{Graphe de captation} : \fbox{$G^{capt} = (T, E^{capt})$} 
  \end{itemize}
  \item $E^{com} = \{ (t, t')\in (T \cup \{ s \})^2 | t\neq t', \Delta_{t, t'} \leq R^{com} \}$
  \begin{itemize}
    \item[$\rightarrow$] $E^{com}_t = \{t'\in T | (t, t')\in E^{com}\}$
    \item[$\rightarrow$] \textbf{Graphe de communication} : \fbox{$G^{com} = (T \cup \{s\}, E^{com})$}
    \item[$\rightarrow$] On note $\mathcal{D}(E^{com})$ l'ensemble des arrêtes orientées contenant les $(t, t')$ et $(t', t)$.
  \end{itemize}
  \item $\forall t\in T,\, \delta_t = \1_{\{\text{capteur sur la cible } t\}}$
  \item $\forall e\in E^{com},\, x_e = \1_{\{e \text{ dans l'arbre de communication}\}}$
\end{bulletlist}

\subsection{Modèle PLNE}

Variables de décision :
\begin{bulletlist}
  \item $\forall t\in T,\, \delta_t = \1_{\text{capteur sur la cible } t}$
  \item $\forall e\in E^{com},\, x_e = \1_{e \text{ dans l'arbre de communication}}$
\end{bulletlist}

On note $\delta_s = 1$.

\begin{minie}|s|[2]
  {\delta, x, f}
  {\sum\limits_{t\in T}\delta_t \label{objectiveReference}}
  {\label{problemReference}}  
  {}%{v^*(\delta, x, f) = }
  \addConstraint{\sum\limits_{t'\in E^{capt}_t} \delta_{t'}}{\geq k, }{\forall t\in T}\label{k-connex}
  \addConstraint{\sum\limits_{e\in E^{com}}x_e}{= n,}{}
  \addConstraint{x_e}{\leq \delta_t, }{\forall e=(t, t')\in E^{com}}
  \addConstraint{x_e}{\leq \delta_{t'}, }{\forall e=(t, t')\in E^{com}}
  \addConstraint{\sum\limits_{t'\in E^{com}_t} y_{(t',t)} - \sum\limits_{t'\in E^{com}_t} y_{(t,t')}}{= \delta_t,}{\forall t\in T}
  \addConstraint{\sum\limits_{t\in E^{com}_s} y_{(s,t)} - \sum\limits_{t\in E^{com}_s} y_{(t,s)}}{= \sum\limits_{t\in T}\delta_t, \quad}{}
  \addConstraint{y_{(t, t')}}{\leq n\times x_e, }{\forall e=(t, t')\in E^{com}}
  \addConstraint{y_{(t', t)}}{\leq n\times x_e, }{\forall e=(t, t')\in E^{com}}
  \addConstraint{\delta_t}{\in\{0, 1\}, }{\forall t\in T}
  \addConstraint{x_e}{\in\{0, 1\}, }{\forall e\in E^{com}}
  \addConstraint{y_a}{\geq 0, }{\forall a\in \mathcal{D}(E^{com})}
\end{minie}

\begin{bulletlist}
  \item k-connexité : contrainte \ref{k-connex}
\end{bulletlist}

\subsection{Relaxation linéaire}

\section{Heuristiques}

\subsection{heuristique 1}

\begin{algorithm}[H]
  \caption{Heuristique par captations de coût minimal successives}
  \label{Heuristique par captations de coût minimal successives}
  \begin{algorithmic}
      \STATE \textbf{Input:}
      \STATE $(T, G^{capt}, G^{com}, k)$ : une instance du problème
      \STATE $L$ : une liste contenant $k$ fois chaque cible de $T$ dans un ordre quelconque
      \STATE
      \STATE \textbf{Output:} $N \subset T$ : le sous-ensemble de cibles sur lesquelles placer des capteurs
      \STATE
      \STATE $N = \emptyset$
      \FORALL{$t \in L$}
          \STATE Choisir $t'$ tel que $(t, t') \in E^{capt}$ et qui minimise le plus court chemin vers $s \cup N$ dans $G^{com}$
          \STATE $E^{capt} = E^{capt} \setminus (t, t')$
          \STATE Ajouter à $N$ les cibles situées sur le plus court chemin de $t'$ à $s \cup N$ dans $G^{com}$
      \ENDFOR
      \STATE Renvoyer $N$
  \end{algorithmic}
\end{algorithm}

\subsection{heuristique 2}

\begin{algorithm}[H]
  \caption{Heuristique gloutonne avec relation d'ordre sur les cibles}
  \label{Heuristique gloutonne avec relation d'ordre sur les cibles}
  \begin{algorithmic}
      \STATE \textbf{Input:}
      \STATE $(T, G^{capt}, G^{com},k)$ : une instance du problème
      \STATE $\prec$ : une relation d'ordre sur les cibles
      \STATE
      \STATE \textbf{Output:} $N \subset T$ : le sous-ensemble de cibles sur lesquelles placer des capteurs
      \STATE
      \STATE $N = \emptyset$
      \STATE $Q = \{t \in T | (s, t) \in E^{com}\}$
      \STATE $U = \emptyset$
      \WHILE{Toutes les cibles n'ont pas au moins $k$ voisins appartenant à $N$ dans $G^{capt}$}
          \IF{$Q \setminus U \neq \emptyset$}
              \STATE Choisir $t$ dans $Q \setminus U$ maximal pour la relation d'ordre $\prec$
              \IF{$t$ a au moins un voisin dans $G^{capt}$ ayant moins de $k$ voisins dans $N$}
                  \STATE $Q  = Q \setminus \{t\}$
                  \STATE $Q  = Q \cup \{t' | (t, t') \in E^{com}\}$
                  \STATE $N  = N \cup \{t\}$
              \ELSE
                  \STATE $U  = U \cup \{t\}$
              \ENDIF
          \ELSE 
              \STATE Choisir $t$ dans $Q \cap U$ maximal pour la relation d'ordre $\prec$
              \STATE $Q  = Q \setminus \{t\}$
              \STATE $Q  = Q \cup \{t' | (t, t') \in E^{com}\}$
              \STATE $N  = N \cup \{t\}$
          \ENDIF
      \ENDWHILE
      \STATE Renvoyer $N$
  \end{algorithmic}
\end{algorithm}

\section{Métaheuristiques}

\subsection{Structure de voisinage}

\subsection{Recuit simulé}

\subsection{Recherche à voisinages multiples}

\end{document}
